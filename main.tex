\documentclass[9pt,a4paper]{extarticle}
\usepackage[T1]{fontenc}
\usepackage{tabto}
\usepackage{float}
\catcode`\_=12

\usepackage{f1000_styles}
\usepackage{verbatim}
%% Default: numerical citations
\usepackage[numbers]{natbib}
%% Uncomment this lines for superscript citations instead
% \usepackage[super]{natbib}
%% Uncomment these lines for author-year citations instead
% \usepackage[round]{natbib}
% \let\cite\citep

\newenvironment{allintypewriter}{\ttfamily}{\par}



\begin{document}
\pagestyle{front}
%%%%%%%%%%%%%%%%%%%%%%%%%%%%%%%%%%%%%%%%%%%%%%%%%%%%%%%%%%%%%%%%%%%%%%%%%%%%%%%%
\title{pysradb: A Python package to query next-generation ~ sequencing metadata
and data from NCBI Sequence Read Archive}
%\titlenote{The title should be detailed enough for someone to know whether the
%article would be of interest to them, but also concise. Please ensure the 
%broadness and claims within the title are appropriate to the content of the
%article itself.}

\author[]{Saket Choudhary}
\affil[]{Computational Biology and Bioinformatics\\
University Of Southern California, Los Angeles, CA 900089, USA.}
\affil[]{email: skchoudh@usc.edu}


\maketitle
\thispagestyle{front}


\begin{abstract}
%Is the rationale for developing the new software tool clearly explained?

%Is the description of the software tool technically sound?

%Are sufficient details of the code, methods and analysis (if applicable) provided to allow replication of the software development and its use by others?

%Is sufficient information provided to allow interpretation of the expected output datasets and any results generated using the tool?

%Are the conclusions about the tool and its performance adequately supported by the findings presented in the article?

%%%%%%%%%%%%%%%%%%%%%%%%%%%%%%%%%%%%%%%%%%%%%%%%%%%%%%%%%%%%%%%%%%%%%%%%%%%%%%%%
NCBI’s Sequence Read Archive (SRA) is the primary archive of next-generation 
sequencing datasets. SRA makes metadata and raw sequencing data available to the
research community to encourage reproducibility, and to provide avenues for 
testing novel hypotheses on publicly available data. However, methods to
programmatically access this data are limited. We introduce a
Python package \texttt{pysradb} that provides a collection of  command
line methods to query and download metadata and data from SRA utilizing the
curated metadata database available through the SRAdb project. We demonstrate 
the utility of pysradb for searching and downloading SRA datasets.
%\texttt{pysradb} streamlines the process of retriving data and metadata

%Abstracts should be up to 300 words and provide a succinct summary of the
%article. Although the abstract should explain why the article might be 
% interesting, care should be taken not to inappropriately over-emphasise the 
%importance of the work described in the article. Citations should not be used
%in the abstract, and the use of abbreviations should be minimized.

\end{abstract}

\section*{Keywords}
bioinformatics, SRA, NGS, NCBI, metadata, GEO, Python
%Please list up to eight keywords to help readers interested in your article find it more easily.



\clearpage
\pagestyle{main}
\section*{Introduction}
%%%%%%%%%%%%%%%%%%%%%%%%%%%%%%%%%%%%%%%%%%%%%%%%%%%%%%%%%%%%%%%%%%%%%%%%%%%%%%%%
Several projects have made efforts to analyze and publish summaries of DNA-seq \cite{macarthur2012systematic} and RNA-seq
\cite{lachmann2018massive, collado2017reproducible} datasets. Obtaining metadata
and raw data from NCBI's Sequencing Read Archive (SRA) \cite{leinonen2010sequence}
is often the first step towards re-analyzing public next-generation sequencing 
datasets to compare them to private data or test a novel hypothesis. NCBI's SRA
toolkit \cite{ncbisratoolit} provides utility methods to download raw sequencing
data, while the metadata can be obtained by querying the website or through the
Entrez \texttt{efetch} command line utility \cite{kans2018entrez}. Most workflows analyzing
public data rely on first searching for relevant keywords in the metadata either
through the command line utility or the website and then downloading these. A
more streamlined workflow can enable doing both these steps at once.


%%%%%%%%%%%%%%%%%%%%%%%%%%%%%%%%%%%%%%%%%%%%%%%%%%%%%%%%%%%%%%%%%%%%%%%%%%%%%%%%
%% MORE?
% Reorganize implementation
% developed and maintained
In order to make querying both metadata and data more precise and robust, SRAdb
\cite{zhu2013sradb} project provides a frequently updated SQLite database 
containing all the metadata parsed from SRA. \texttt{SRAdb} tracks the five main
data objects in SRA's metadata: submission, study, sample, experiment and run.
These are mapped to five different relational database tables that are made
available in the SQLite file. The metadata semantics in the file remain as they are on
SRA.
%SRAdb's metadata semantics remain as they are in SRA with minor changes in the field names to improve the usability of SQL queries. 
The accompanying package \texttt{SRAdb} \cite{zhudavissradb} made available in the R programming language \cite{rprog}
provides a convenient framework to handle metadata query and raw data downloads by utilizing the SQLite database. 
Though powerful, SRAdb requires the end user to be familiar 
with the R programming language and does not provide a command-line interface for
querying or downloading operations.
%%%%%%%%%%%%%%%%%%%%%%%%%%%%%%%%%%%%%%%%%%%%%%%%%%%%%%%%%%%%%%%%%%%%%%%%%%%%%%%%


\texttt{pysradb} package builds upon the principles of \texttt{SRAdb} and 
provides a simple and user-friendly command-line interface for querying metadata and downloading
datasets from SRA. It obviates the need for the
user to be familiar with any programming language for querying and
downloading datasets from SRA. Additionally, it provides utility 
functions that will further help a user perform more granular queries, that
are often required when dealing with multiple datasets at large scale. By enabling both metadata
search and download operations at the command-line, \texttt{pysradb} aims to bridge the gap in
seamlessly retrieving public sequencing datasets and the associated metadata.

\texttt{pysradb} is written in Python (Python Software Foundation, 
https://www.python.org/) \cite{vanRossum:2011:PLR:2011965} and is currently developed
on Github under the open-source BSD 3-Clause License. In order to simplify the installation
procedure for the end-user, it is also available for download through PyPI (https://pypi.org/project/pysradb/)
and bioconda \cite{gruning2018bioconda} (https://bioconda.github.io/recipes/pysradb/README.html)

\section*{Methods}

\subsection*{Implementation}
%%%%%%%%%%%%%%%%%%%%%%%%%%%%%%%%%%%%%%%%%%%%%%%%%%%%%%%%%%%%%%%%%%%%%%%%%%%%%%%%
\texttt{pysradb} is implemented in Python and uses
\texttt{pandas} \cite{mckinney-proc-scipy-2010} for data frame based operations. 
Since, downloading datasets can often take long time, \texttt{pysradb} displays 
progress for long haul tasks using \texttt{tqdm}
\cite{casper_da_costa_luis_2018_1211527}. The metadata information is read in the
form of a SQLite \cite{about_sqlite} database made available by SRAdb
\cite{zhu2013sradb}. 
%\texttt{pysradb} also supports accessing metadata 
%information from Gene Expression Omnibus (GEO) \cite{edgar2002gene,
%barrett2012ncbi} through the SQLite database made available through the
%GEOmetadb project \cite{zhu2008geometadb}.

%%%%%%%%%%%%%%%%%%%%%%%%%%%%%%%%%%%%%%%%%%%%%%%%%%%%%%%%%%%%%%%%%%%%%%%%%%%%%%%%
\texttt{pysradb} can be run on either Linux or Mac based operating systems. It
is implemented in Python programming language, has minimal dependencies and can
be easily installed using either \texttt{pip} or \texttt{conda} based package
manager via the \texttt{bioconda} \cite{gruning2018bioconda} channel. 

\texttt{pysradb}'s development primarily happen on Github
and the code is tested continuously using Travis CI webhook. This monitors
all incoming pull requests and commits to the master branch. The testing
happens on Python version 3.5, 3.6, and 3.7 on an Ubuntu 16.04 LTS virtual
machine while the testing webhooks on the bioconda channel provides
additional testing on Mac based systems. Nevertheless, \texttt{pysradb} should
run on most Unix derivatives.

Each sub-command of \texttt{pysradb} contains a self-contained help string,
that describes its purpose and usage example. The help text can be accessed
by passing the `--help' flag. There is also additional documentation available
for the sub-commands on the project's website (https://saketkc.github.io/pysradb).
We also provide example Jupyter \cite{Kluyver:2016aa} notebooks (https://github.com/saketkc/pysradb/tree/master/notebooks)
that demonstrate the functionality of the Python API. 

%For software tool papers, this section should address how the tool works and any relevant technical details required for implementation of the tool by other developers.  

%\subsection*{Operation}
%This part of the methods should include the minimal system requirements needed to run the software and an overview of the workflow for the tool for users of the tool.

%%%%%%%%%%%%%%%%%%%%%%%%%%%%%%%%%%%%%%%%%%%%%%%%%%%%%%%%%%%%%%%%%%%%%%%%%%%%%%%%



\section*{Use Cases} % Optional - only if NO new datasets are included
%%%%%%%%%%%%%%%%%%%%%%%%%%%%%%%%%%%%%%%%%%%%%%%%%%%%%%%%%%%%%%%%%%%%%%%%%%%%%%%%
% TOO informal
% thisby
\texttt{pysradb} provides a chain of sub-commands for retrieving metadata, converting
one accession to other and downloading. Each sub-command is designed to perform
a single operation by default while additional operations can be performed by passing 
additional flags. In the following section we demonstrate some of the use cases
of these sub-commands.
%The primary use case of \texttt{pysradb} is in automated download of an entire
%SRA project. 

\texttt{pysradb} uses \texttt{SRAmetadb.sqlite}, a SQLite file produced and made available by SRAdb 
\cite{zhu2013sradb} project. The file itself can be downloaded using
\texttt{pysradb} as:\\

\begin{allintypewriter}
\$ pysradb srametadb\\
\end{allintypewriter}

\texttt{SRAmetadb.sqlite} file is required for all other operations supported
by \texttt{pysradb}. This file is required for all the sub-commands to function
By default, \texttt{pysradb} assumes that by default the file is located in the current
working directory. Alternatively, it can supplied as `--db path/to/SRAmetadb.sqlite' argument.
The examples here were run using \texttt{SRAmetadb.sqlite} with schema version 1.0 and 
creation timestamp	2019-01-25 00:38:19.

\subsection*{Search}

%%%%%%%%%%%%%%%%%%%%%%%%%%%%%%%%%%%%%%%%%%%%%%%%%%%%%%%%%%%%%%%%%%%%%%%%%%%%%%%%
Consider a case where a user is looking for Ribo-seq \cite{Ingolia2009} public datasets
on SRA. These datasets will often have `ribosome profiling' appearing in
the abstract or sample description. We can search for such projects using
the `search' sub-command:\\

\begin{allintypewriter}
\$ pysradb search `"ribosome profiling"' | head
\begin{table}[H]
    \begin{tabular}{llll}
        study_accession & experiment_accession & sample_accession & run_accession\\
        DRP003075 & DRX019536 & DRS026974 & DRR021383\\ 
        DRP003075 & DRX019537 & DRS026982 & DRR021384\\
        DRP003075 & DRX019538 & DRS026979 & DRR021385\\
        DRP003075 & DRX019540 & DRS026984 & DRR021387\\
        DRP003075 & DRX019541 & DRS026978 & DRR021388\\
        DRP003075 & DRX019543 & DRS026980 & DRR021390\\
        DRP003075 & DRX019544 & DRS026981 & DRR021391\\
        ERP013565 & ERX1264364 & ERS1016056 & ERR1190989  
    \end{tabular}
\end{table}
\end{allintypewriter}

The result here lists down all relevant `ribosome profiling' projects.

\subsection*{Getting metadata for a project (SRP)}

Each SRA project (SRP) on NCBI's SRA consists of one or multiple 
experiments (SRX) which are sequenced as one or multiple runs (SRR). Each
experiment in turn is carried on a biological sample (SRS). 

\texttt{pysradb metadata} can be to obtain all the experiment, sample, and run accessions associated
with a project (SRP) as:\\

\begin{allintypewriter}
\$ pysradb metadata SRP098789 | head
\begin{table}[H]
    \begin{tabular}{llll}
study_accession & experiment_accession & sample_accession & run_accession\\
SSRP010679 & SRX118285 & SRS290854 & SRR403882\\
SRP010679 & SRX118286 & SRS290855 & SRR403883\\
SRP010679 & SRX118287 & SRS290856 & SRR403884\\
SRP010679 & SRX118288 & SRS290857 & SRR403885\\
SRP010679 & SRX118289 & SRS290858 & SRR403886\\
SRP010679 & SRX118290 & SRS290859 & SRR403887\\
SRP010679 & SRX118291 & SRS290860 & SRR403888\\
SRP010679 & SRX118292 & SRS290861 & SRR403889\\
SRP010679 & SRX118293 & SRS290862 & SRR403890\\
SRP010679 & SRX118294 & SRS290863 & SRR403891\\
SRP010679 & SRX118295 & SRS290864 & SRR403892\\
SRP010679 & SRX118296 & SRS290865 & SRR403893
    \end{tabular}
\end{table}
\end{allintypewriter}

However, this information by itself is often not complete. 
We require detailed metadata associated with each sample to perform any downstream 
analysis. For example, the assays used for different samples and the corresponding treatment conditions.
This can be done by supplying the `--desc' flag:\\

\begin{allintypewriter}
\$ pysradb metadata SRP010679 --desc | head -5
\begin{table}[H]
    \begin{tabular}{llllp{40mm}}
study_accession & experiment_accession & sample_accession & run_accession & sample_attribute\\
SRP010679 & SRX118285 & SRS290854 & SRR403882 & source_name: PC3 human prostate cancer cells || cell line: PC3 || sample type: polyA RNA || treatment: vehicle\\
SRP010679 & SRX118286 & SRS290855 & SRR403883 & source_name: PC3 human prostate cancer cells || cell line: PC3 || sample type: ribosome protected RNA || treatment: vehicle\\
SRP010679 & SRX118287 & SRS290856 & SRR403884 & source_name: PC3 human prostate cancer cells || cell line: PC3 || sample type: polyA RNA || treatment: rapamycin\\
SRP010679 & SRX118288 & SRS290857 & SRR403885 & source_name: PC3 human prostate cancer cells || cell line: PC3 || sample type: ribosome protected RNA || treatment: rapamycin\\
    \end{tabular}
\end{table}
\end{allintypewriter}
\begin{comment}
SRP010679 & SRX118289 & SRS290858 & SRR403886 & source_name: PC3 human prostate cancer cells || cell line: PC3 || sample type: polyA RNA || treatment: PP242\\
SRP010679 & SRX118289 & SRS290858 & SRR403886 & source_name: PC3 human prostate cancer cells || cell line: PC3 || sample type: polyA RNA || treatment: PP242\\
SRP010679 & SRX118290 & SRS290859 & SRR403887 & source_name: PC3 human prostate cancer cells || cell line: PC3 || sample type: ribosome protected RNA || treatment: PP242\\
SRP010679 & SRX118291 & SRS290860 & SRR403888 & source_name: PC3 human prostate cancer cells || cell line: PC3 || sample type: polyA RNA || treatment: vehicle\\
SRP010679 & SRX118292 & SRS290861 & SRR403889 & source_name: PC3 human prostate cancer cells || cell line: PC3 || sample type: ribosome protected RNA || treatment: vehicle\\
    %\end{tabular}
%\end{table}
%\end{allintypewriter}
\end{comment}
This can be further expanded to reveal the data in `sample\_attribute' column into
separate columns via `--expand' flag. This is most useful for samples that have treatment or cell type associated
metadata available.\\

\begin{allintypewriter}
\$ pysradb metadata SRP010679 --desc --expand
\begin{table}[H]
    \begin{tabular}{lllll}
    ... [truncated]\\
run_accession & cell_line & sample_type &  source_name &  treatment\\
SRR403882 & pc3 & polya rna & pc3 human prostate cancer cells & vehicle\\
SRR403883 & pc3 & ribosome protected rna & pc3 human prostate cancer cells & vehicle\\
SRR403884 & pc3 & polya rna & pc3 human prostate cancer cells & rapamycin\\
SRR403885 & pc3 & ribosome protected rna & pc3 human prostate cancer cells & rapamycin\\
SRR403886 & pc3 & polya rna & pc3 human prostate cancer cells & pp242\\
SRR403887 & pc3 & ribosome protected rna & pc3 human prostate cancer cells & pp242\\
SRR403888 & pc3 & polya rna & pc3 human prostate cancer cells & vehicle\\
SRR403889 & pc3 & ribosome protected rna & pc3 human prostate cancer cells & vehicle\\
SRR403890 & pc3 & polya rna & pc3 human prostate cancer cells & rapamycin\\
SRR403891 & pc3 & ribosome protected rna & pc3 human prostate cancer cells & rapamycin\\
SRR403892 & pc3 & polya rna & pc3 human prostate cancer cells & pp242\\
SRR403893 & pc3 & ribosome protected rna & pc3 human prostate cancer cells & pp242\\
 \end{tabular}
\end{table}
\end{allintypewriter}

Any SRA project might consist of experiments involving multiple assay types.
The assay associated with any project can be obtained by providing
\texttt{--assay} flag:\\

\begin{allintypewriter}
\$ pysradb metadata SRP000941 --assay  | tr -s '  ' | cut -f5 -d ' ' | tail -n +2 | sort | uniq -c
\begin{table}[H]
    \begin{tabular}{ll}
        999  & Bisulfite-Seq\\
        768 & ChIP-Seq\\
        121 & OTHER\\
        353 & RNA-Seq\\
        28 & WGS\\
    \end{tabular}
\end{table}
\end{allintypewriter}


%NCBI's  sratoolkit \cite{ncbisratoolit} allows downloading raw data per sequencing run. 
%However, it's often an an entire experiment (SRX) that is required rather than
%the individual runs (SRRs). 


\subsection*{Getting SRPs from GSE}

The Gene Expression Omnibus database (GEO) \cite{barrett2012ncbi} is NCBI's data repository
for functional genomics data that accepts array and sequence-based data from gene
profiling experiments. For sequence-data, the corresponding raw files are deposited
to the SRA. GEO assigns its datasets accession (GSE) that are linked to their corresponding
accession on the SRA (SRP). It is often required to interpolate between the two accessions.
\texttt{gse-to-srp} sub-command allows converting GSE to SRP:\\

\begin{allintypewriter}
\$ pysradb gse-to-srp GSE24355 GSE25842
\begin{table}[H]
    \begin{tabular}{ll}
    study_alias & study_accession\\
    GSE24355 & SRP003870\\
    GSE25842 & SRP005378\\
    \end{tabular}
\end{table}
\end{allintypewriter}

It can further be expanded to obtain the corresponding experiment and
run accessions:\\

\begin{allintypewriter}
\$ pysradb gse-to-srp --detailed --expand GSE100007 | head
\begin{table}[H]
    \begin{tabular}{llllll}
study_alias & study_accession & experiment_accession & sample_accession & experiment_alias & sample_alias\\
GSE100007 & SRP109126 & SRX2916198 & SRS2282390 & GSM2667747 & GSM2667747\\
GSE100007 & SRP109126 & SRX2916199 & SRS2282391 & GSM2667748 & GSM2667748\\
GSE100007 & SRP109126 & SRX2916200 & SRS2282392 & GSM2667749 & GSM2667749\\
GSE100007 & SRP109126 & SRX2916201 & SRS2282393 & GSM2667750 & GSM2667750\\
GSE100007 & SRP109126 & SRX2916202 & SRS2282394 & GSM2667751 & GSM2667751\\
GSE100007 & SRP109126 & SRX2916203 & SRS2282395 & GSM2667752 & GSM2667752\\
GSE100007 & SRP109126 & SRX2916204 & SRS2282396 & GSM2667753 & GSM2667753\\ 
GSE100007 & SRP109126 & SRX2916205 & SRS2282397 & GSM2667754 & GSM2667754\\
GSE100007 & SRP109126 & SRX2916206 & SRS2282400 & GSM2667755 & GSM2667755\\
    \end{tabular}
\end{table}
\end{allintypewriter}

\subsection*{Getting a list of GEO experiments for a GEO study}
Any GEO study (GSE) will involve a collection of experiments (GSM). We can obtain
an entire list of experiments corresponding to the study using the
\texttt{gse-to-gsm} sub-command from \texttt{pysradb}:\\
\begin{allintypewriter}
\$ pysradb gse-to-gsm GSE41637 | head
\begin{table}[H]
    \begin{tabular}{ll}
        study_alias & experiment_alias\\
        GSE41637  &  GSM1020640_1\\
        GSE41637  &  GSM1020641_1\\
        GSE41637  &  GSM1020642_1\\
        GSE41637  &  GSM1020643_1\\
        GSE41637  &  GSM1020644_1\\
        GSE41637  &  GSM1020645_1\\
        GSE41637  &  GSM1020646_1\\
        GSE41637  &  GSM1020647_1\\
        GSE41637  &  GSM1020648_1\\ 
    \end{tabular}
\end{table}
\end{allintypewriter}

However, just a list of GSM accessions is not useful if one is performing any downstream
analysis which essentially requires more detailed information about the metadata
associated with each experiment. This relevant metadata associated with each sample can be obtained by providing \texttt{gse-to-gsm} additional flags:\\

\begin{allintypewriter}
\$ pysradb gse-to-gsm --desc GSE41637 | head
\begin{table}[H]
    \begin{tabular}{lll}
        study_alias & experiment_alias & sample_attribute\\
        GSE41637 & GSM1020640_1 &  source_name: mouse_brain || strain: DBA/2J || tissue: brain\\
        GSE41637 & GSM1020641_1 &  source_name: mouse_colon || strain: DBA/2J || tissue: colon\\
        GSE41637 & GSM1020642_1 &  source_name: mouse_heart || strain: DBA/2J || tissue: heart\\
        GSE41637 & GSM1020643_1 &  source_name: mouse_kidney || strain: DBA/2J || tissue: kidney\\
        GSE41637 & GSM1020644_1 &  source_name: mouse_liver || strain: DBA/2J || tissue: liver\\
        GSE41637 & GSM1020645_1 &  source_name: mouse_lung || strain: DBA/2J || tissue: lung\\
        GSE41637 & GSM1020646_1 &  source_name: mouse_skm || strain: DBA/2J || tissue: skeletal muscle\\
        GSE41637 & GSM1020647_1 &  source_name: mouse_spleen || strain: DBA/2J || tissue: spleen\\
        GSE41637 & GSM1020648_1 &  source_name: mouse_testes || strain: DBA/2J || tissue: testes\\
    \end{tabular}
\end{table}
\end{allintypewriter}

The metadata information can then be parsed from the \texttt{sample\_attribute} column.
To obtain more structured metadata, we can use an additional flag `\texttt{--expand}':\\

\begin{allintypewriter}
\$ pysradb gse-to-gsm --desc --expand GSE41637 | head
\begin{table}[H]
    \begin{tabular}{lllll}
        study_alias & experiment_alias & source_name &  strain & tissue\\
        GSE41637 & GSM1020640_1 &  mouse_brain &  dba/2j & brain\\
        GSE41637 & GSM1020641_1 &  mouse_colon &  dba/2j & colon\\
        GSE41637 & GSM1020642_1 &  mouse_heart &  dba/2j & heart\\
        GSE41637 & GSM1020643_1 &  mouse_kidney & dba/2j & kidney\\
        GSE41637 & GSM1020644_1 &  mouse_liver &  dba/2j & liver\\
        GSE41637 & GSM1020645_1 &  mouse_lung & dba/2j & lung\\
        GSE41637 & GSM1020646_1 &  mouse_skm & dba/2j & skeletal muscle\\
    \end{tabular}
\end{table}
\end{allintypewriter}

\subsection*{Getting SRR from GSM}
\texttt{gsm-to-srr} allows conversion from GEO experiments (GSM) to
SRA runs (SRR):\\

\begin{allintypewriter}
\$ pysradb gsm-to-srr  GSM1020640 GSM1020646
\begin{table}[H]
    \begin{tabular}{lll}
experiment_alias & run_accession\\
GSM1020640_1 & SRR594393\\
GSM1020646_1 & SRR594399\\
\end{tabular}
\end{table}
\end{allintypewriter}

\subsection*{Downloading SRA datasets}

\texttt{pysradb} enables seemless downloads from NCBI's SRA. It organizes
the downloaded data following NCBI's hiererachy: `SRP => SRX => SRR' of storing data.
Each `SRP' (project) has multiple `SRX' (experiments) and each `SRX' in turn 
has multiple `SRR' (runs).  Multiple projects can be downloaded at once using
the \texttt{download} sub-command:\\


\begin{allintypewriter}
\$  pysradb download -p SRP003870 -p SRP005378
\end{allintypewriter}

\texttt{download} also allows Unix pipes based inputs. Consider our previous
example of the project SRP000941 with different assays. However, we want to be able
to download only `RNA-seq' samples. We can do this by subsetting the metadata
output for only `RNA-seq' samples:\\


\begin{allintypewriter}
\$ pysradb metadata SRP000941 --assay | grep `study|RNA-Seq' | pysradb download
\end{allintypewriter}



This will only download the `RNA-seq' samples from the project.


%%%%%%%%%%%%%%%%%%%%%%%%%%%%%%%%%%%%%%%%%%%%%%%%%%%%%%%%%%%%%%%%%%%%%%%%%%%%%%%%
%The simplest use case of `pysradb` is when you apriopri know the SRA project ID 
%(SRP) and would simply want to fetch the metadata associated with it. This is
%generally reflected in the `SraRunTable.txt' that you get from NCBI's website.
%Example: https://www.ncbi.nlm.nih.gov/Traces/study/?acc=SRP098789.

%%%%%%%%%%%%%%%%%%%%%%%%%%%%%%%%%%%%%%%%%%%%%%%%%%%%%%%%%%%%%%%%%%%%%%%%%%%%%%%%

%%%%%%%%%%%%%%%%%%%%%%%%%%%%%%%%%%%%%%%%%%%%%%%%%%%%%%%%%%%%%%%%%%%%%%%%%%%%%%%%
\begin{comment}
Once you have fetched the metadata and made sure, this is the project you were 
looking for, you would want to download everything at once. NCBI follows this 
hiererachy: `SRP => SRX => SRR'. Each `SRP' (project) has multiple `SRX` 
(experiments) and each `SRX' in turn has multiple `SRR` (runs) inside it.
We want to mimick this hiereachy in our downloads. The reason to do that is simple:
in most cases you care about `SRX' the most, and would want to "merge" your SRRs
in one way or the other. Having this hierearchy ensures your downstream code
can handle such cases easily, without worrying about which runs (SRR) need to
be merged.


\begin{allintypewriter}
\$ pysradb download -p SRP063852 
\end{allintypewriter}

Often, you need to process only a smaller set of samples from a project (SRP).
Consider the project SRP000941 which has data spanning four assays. But, we might be 
only interested in analyzing the `RNA-seq' samples and would just want to download
that subset. This can be done simply using the following command:



%%%%%%%%%%%%%%%%%%%%%%%%%%%%%%%%%%%%%%%%%%%%%%%%%%%%%%%%%%%%%%%%%%%%%%%%%%%%%%%%
\begin{allintypewriter}
\$ pysradb metadata SRP000941 --assay $|$ grep `study$|$RNA-Seq' $|$ pysradb download
\end{allintypewriter}

\end{comment}


\begin{comment}

\section*{Discussion} % Optional - only if novel data or analyses are included
%%%%%%%%%%%%%%%%%%%%%%%%%%%%%%%%%%%%%%%%%%%%%%%%%%%%%%%%%%%%%%%%%%%%%%%%%%%%%%%%
This section is only required if the paper includes novel data or analyses, and 
should be written in the same style as a traditional discussion section. Please 
include a brief discussion of allowances made (if any) for controlling bias or unwanted sources of variability, and the limitations of any novel datasets.


\section*{Conclusions} % Optional - only if novel data or analyses are included
This section is only required if the paper includes novel data or analyses, and should be written as a traditional conclusion.
\end{comment}

\section*{Summary} % Optional - only if NO new datasets are included
%%%%%%%%%%%%%%%%%%%%%%%%%%%%%%%%%%%%%%%%%%%%%%%%%%%%%%%%%%%%%%%%%%%%%%%%%%%%%%%% 
%The need to perform metaanalysis warrants need of more stream lined ways to obtain
%data. NCBI provides access to a huge set of datasets across 
\texttt{pysradb} provides a command-line interface to query metadata and download
sequencing datasets from NCBI's SRA. It enables seamless retrieval of metadata
and conversion between different accessions. \texttt{pysradb} is written in Python 3
and is available on Linux and Mac OS. The source code is hosted on Github and licensed
under BSD 3-clause license. It is available for installation through PyPI and bioconda.


\begin{comment}


\section*{Data availability} % Optional - only if novel data or analyses are included
Please add details of where any datasets that are mentioned in the paper, and that have not have not previously been formally published, can be found.  If previously published datasets are mentioned, these should be cited in the references, as per usual scholarly conventions.
\end{comment}

\section*{Software availability}
Software and source code available from: https://github.com/saketkc/pysradb

Documentation available at: https://saketkc.github.io/pysradb

Archived source code at time of publication: https://doi.org/10.5281/zenodo.2579446

Software license: BSD 3-Clause

\section*{Author Contributions}
S.C. designed the project, implemented the package, and wrote the manuscript.

\section*{Competing interests}
No competing interests were disclosed.

\section*{Grant information}
The author declared that no grants were involved in supporting this work.

\section*{Acknowledgments}
The author thanks Meng Zhou, Rishvanth Prabakar, and Amal Thomas at the University Of Southern California
and Shweta Ramdas at the University of Pennsylvania for helpful discussions and comments
on the software and manuscript. The author acknowledges support from USC Provost Graduate Research Fellowship. 

\begin{comment}
%%%%%%%%%%%%%%%%%%%%%%%%%%%%%%%%%%%%%%%%%%%%%%%%%%%%%%%%%%%%%%%%%%%%%%%%%%%%%%%%
This section should acknowledge anyone who contributed to the research or the
article but who does not qualify as an author based on the criteria provided earlier
(e.g. someone or an organization that provided writing assistance). Please state how
they contributed; authors should obtain permission to acknowledge from all those
mentioned in the Acknowledgments section.

Please do not list grant funding in this section.
\end{comment}

{\small\bibliographystyle{unsrtnat}
\bibliography{bibliography}}

\begin{comment}

5. Main Body
Software Tool Articles typically contain the following sections:
Introduction
Methods, providing details of Implementation and Operation
Results (optional)
Use Cases (optional)
Discussion/Conclusions
The Introduction should provide context as to why the software tool was developed and what need it addresses. It is good scholarly practice to mention previously developed tools that address similar needs, and why the current tool is needed.
The Methods should include a subsection on Implementation describing how the tool works and any relevant technical details required for implementation; and a subsection on Operation, which should include the minimal system requirements needed to run the software and an overview of the workflow.
A Results section is only required if the paper includes novel data or analyses, and should be written as a traditional results section.
Please include a section on Use Cases if the paper does not include novel data or analyses. Examples of input and output files should be provided with some explanatory context. Any novel or complex variable parameters should be explained in sufficient detail to enable users to understand and use the tool's functionality.
A Discussion (e.g. if the paper includes novel data or analyses) or Conclusions should include a brief discussion of allowances made (if any) for controlling bias or unwanted sources of variability, and the limitations of any novel datasets.
Reproducibility: F1000Research is committed to serving the research community by ensuring that all articles include sufficient information to allow others to reproduce the work. With this in mind, Methods sections should provide sufficient details of the materials and methods used so that the work can be repeated by others. The section should also include a brief discussion of allowances made (if any) for controlling bias or unwanted sources of variability. Any limitations of the datasets should be discussed.
When antibodies are used, the species in which the antibody was raised, the manufacturing company or source laboratory, the catalogue or reference number, and whether it is a polyclonal or monoclonal antibody should be included. In addition, if the antibody has been previously validated, a reference to the validation study should be included. If the antibody has not been validated, full details of the dilution and use of the antibody should be given in the Methods section.
We encourage authors to add Research Resource Identifiers (RRIDs) to their article in order to unambiguously identify the following types of resources: antibodies, genetically modified organisms, software tools, data, databases and services. More information on this project is available from the Resource Identification Initiative and RRIDs can be obtained from the portal.
Where applicable, we also encourage authors to deposit a step-by-step description of their protocols on protocols.io, where they obtain a persistent digital object identifier (DOI), which can be included in the Methods section of the article, using https://doi.org/10.17504/protocols.io.[PROTOCOL DOI] as the format (e.g. https://doi.org/10.17504/protocols.io.hrkb54w). Authors should note that the protocol is only made public once they select “Publish” on protocols.io.

\end{comment}

% See this guide for more information on BibTeX:
% http://libguides.mit.edu/content.php?pid=55482&sid=406343

% For more author guidance please see:
% https://f1000research.com/for-authors/article-guidelines/software-tool-articles


% When all authors are happy with the paper, use the 
% ‘Submit to F1000RESEARCH' button from the menu above
% to submit directly to the open life science journal F1000Research.

% Please note that this template results in a draft pre-submission PDF document.
% Articles will be professionally typeset when accepted for publication.

% We hope you find the F1000Research Overleaf template useful,
% please let us know if you have any feedback using the help menu above.


\end{document}